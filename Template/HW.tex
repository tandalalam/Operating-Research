\documentclass[12pt]{article}
\usepackage{HomeWorkTemplate}
\usepackage{circuitikz}
\usepackage[shortlabels]{enumitem}
\usepackage{hyperref}
\usepackage{tikz}
\usepackage{xepersian}
%\usepackage{lineno}
%\usepackage{xcolor}
%\usepackage[colorlinks]{hyperref}
%\usepackage{color}
\usetikzlibrary{arrows,automata}
\usetikzlibrary{circuits.logic.US}
\settextfont{HM XNiloofar}
\setdigitfont{HM XNiloofar}
\usepackage{changepage}
%\usepackage[inline]{enumitem}
\newcounter{problemcounter}
\newcounter{solutioncounter}
%\newcounter{subproblemcounter}
\newcounter{judgmentcounter}
\newcounter{judgmentanswercounter}
\setcounter{problemcounter}{1}
\setcounter{solutioncounter}{1}
\setcounter{judgmentcounter}{1}
\setcounter{judgmentanswercounter}{1}
%\setcounter{subproblemcounter}{1}
\newcommand{\problem}[1]
{
	\subsection*{
		سوال
		\arabic{problemcounter} 
		\stepcounter{problemcounter}
		%\setcounter{subproblemcounter}{1}
		#1
	}
}
\newcommand{\solution}[1]
{
	\subsection*{
		پاسخ سوال 
		\arabic{solutioncounter} 
		\stepcounter{solutioncounter}
		%\setcounter{subproblemcounter}{1}
		#1
	}
}

\newcommand{\judgment}[1]
{
	\subsection*{
		داوری سوال 
		\arabic{judgmentcounter} 
		\stepcounter{judgmentcounter}
		%\setcounter{subproblemcounter}{1}
		#1
	}
}

\newcommand{\judgmentanswer}[1]
{
	\subsection*{
		پاسخ داوری سوال 
		\arabic{judgmentanswercounter} 
		\stepcounter{judgmentanswercounter}
		%\setcounter{subproblemcounter}{1}
		#1
	}
}

%\newcommand{\subproblem}{
%	\textbf{\harfi{subproblemcounter})}\stepcounter{subproblemcounter}
%}



\begin{document}
\handout
{تحقیق در عملیات 1}
{تمرین سری یک}
{مهدی مستانی، شایان طاهری‌جم، سینا کلانترزاده و متین امینی}
{0}
{}
\problem{} 
ترجمه صورت سوال را باید اینجا وارد کنید. اگر سوال چندقسمتی است، از محیط زیر استفاده کنید:
\begin{enumerate}[(a)]
	\item قسمت اول سوال
	\item قسمت دوم سوال
\end{enumerate}
\solution{} پاسخ سوال را باید اینجا وارد کنید. اگر سوال چندقسمتی است، از محیط زیر استفاده کنید:
\begin{enumerate}[(a)]
	\item پاسخ قسمت اول سوال
	\item پاسخ قسمت دوم سوال
\end{enumerate}
برای تعریف یک مساله‌ی برنامه‌ریزی خطی از قالب‌ها و نمونه‌های زیر کمک بگیرید:
\begin{equation*}
\left\{
\begin{array}{cll}
\min_x &  c^T x & \\
\mathrm{s.t.} & A x = b, & \\
& x\ge 0.
\end{array}  \right.
\end{equation*}

\begin{equation*}
\left\{
\begin{array}{cl}
\min_{x} &  2x_1 - x_2 + 4 x_3\\
\mathrm{s.t.} & x_1 + x_2 + x_4 \le 2,\\
& 3 x_2 - x_3 = 5,\\
& x_3 + x_4 \ge 2,\\
& x_1 \ge 0,\\
& x_2 \le 0.
\end{array}  \right.
\end{equation*}

\judgment{}
گروه پاسخ‌دهنده باید این قسمت را خالی رها کند.\\
در بخش داوری سوال، گروه مسئول داوری، باید پاسخ را با دقت بررسی کند و آن را از نظر صحیح بودن، مفهوم بودن، کامل بودن و از نظر نگارشی ارزیابی کند. اگر می‌خواهید ویرایش‌های نگارشی خود را در متن پاسخ اعمال کنید باید آن‌ها را با رنگ قرمز مشخص کنید. برای این منظور، از دستور 
\textcolor{red}{textcolor}
استفاده کنید. نظرات را در دو بخش جداگانه، یعنی 
\textbf{تصحیح‌های نگارشی}
و 
\textbf{تصحیح‌های علمی}
به صورت زیر اعلام کنید:
\begin{itemize}
	\item 
	\textbf{تصحیح‌های نگارشی:}
		\begin{enumerate}
			\item
			\item
		\end{enumerate}
		\item 
	\textbf{تصحیح‌های علمی:}
	\begin{enumerate}
		\item
		\item
	\end{enumerate}
\end{itemize} 
در انتهای بخش داوری، باید در یک پاراگراف، یک گزارش کلی از پاسخ ارائه دهید و اعلام کنید که پاسخ قابل قبول است یا نه (با ارائه دلیل کافی)؟ ویرایش جدی نگارشی نیاز دارد یا نه؟ ویرایش علمی نیاز دارد یا نه؟ 
\judgmentanswer{}
در بخش پاسخ داوری، باید به تمام نظرات گروه داوری، با ارائه دلایل کافی و به صورت محترمانه پاسخ دهید. قضاوت در مورد این که داوری منصفانه است یا خیر، و پاسخ به آن قابل قبول است یا خیر به عهده دستیاران آموزشی است. مدرس درس به عنوان آخرین مرجع، به این فرایند نظارت خواهد داشت. در زمان بازنگری، تصحیح‌ها و به طور کلی هر تغییری در پاسخ خود را باید با رنگ آبی مشخص کنید. برای این منظور، از دستور 
\textcolor{blue}{textcolor}
استفاده کنید.
\end{document}